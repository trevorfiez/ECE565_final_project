\section{Maximum Likelihood and Expectation-Maximization}

\subsection{Maximum Likelihood Equations}
Given Equation \ref{main_pmf}, let the likelihood be
\begin{align*}
L(\mathbb{\theta}) = p(k_1(1), \cdots, k_1(N) | \mathbb{\theta})\ 
\end{align*}
Then, the log-likelihood is
\begin{align*}
\log L(\mathbb{\theta}) &= \sum_{i=1}^{N} \log \left[ \alpha \binom{n}{k_1(i)} p^{k_1(i)} (1-p)^{n - k_1(i)} + (1-\alpha) \binom{n}{k_1(i)} q^{k_1(i)} (1-q)^{n - k_1(i)}  \right] 
\end{align*}

The maximum log-likelihood is given by
\begin{align*}
\hat{\theta} = \argmax_{\theta} \log L(\mathbb{\theta})
\end{align*}
Therefore, the maximum likelihood equations are given by:
%Want to do: $\frac{\delta \log p}{\delta \mathbf{\theta}} = \mathbf{0}$

\begin{align*}
\frac{\delta \log p}{\delta \mathbf{\theta}} &= \begin{bmatrix} 
\sum_{i=1}^{N} \frac{\phi_1(k_1(i)) - \phi_0(k_1(i))}{\alpha \phi_1(k_1(i)) + (1-\alpha)\phi_0(k_1(i))} \\ 
\sum_{i=1}^{N} \frac{\alpha \binom{n}{k_1(i)} \left( k_1(i) p^{k_1(i)-1} (1-p)^{n-k_1(i)} - p^{k_1(i)} (n-k_1(i)) (1-p)^{n-k_1(i)-1} \right) }{\alpha \phi_1(k_1(i)) + (1-\alpha)\phi_0(k_1(i))} \\ 
\sum_{i=1}^{N} \frac{(1-\alpha) \binom{n}{k_1(i)} \left( k_1(i) q^{k_1(i)-1} (1-q)^{n-k_1(i)} - q^{k_1(i)} (n-k_1(i)) (1-q)^{n-k_1(i)-1} \right) }{\alpha \phi_1(k_1(i)) + (1-\alpha)\phi_0(k_1(i))} \end{bmatrix} = \mathbf{0}
\end{align*}

\subsection{EM Algorithm}

Let $\mathbf{k_1} = k_1(1), \cdots, k_1(N)$ be the observed data. 
We introduce membership variables $\mathbf{y} = y(i), \cdots, y(N)$ (hidden data) such that $p(y(i) = c) = \alpha_c$. Because we only have two classes, then 
\begin{align*}
p(y(i) = 1) &= \alpha \\
p(y(i) = 0) &= 1-p(y(i) = 1) = 1 - \alpha
\end{align*}

The joint probability mass function is given by:
\begin{align*}
p(\mathbf{k_1}, \mathbf{y} | \theta ) &= \prod_{i=1}^{N} p(k_1(i), y(i) | \theta ) \\
&= \prod_{i=1}^{N} p(k_1(i) | y(i) ) p(y(i) | \theta )  \\
&= \prod_{i=1}^{N} \prod_{l=1}^{M} (\phi_l(k_1(i) \alpha_l)) ^ {I(y(i)=l)}
\end{align*}

Let $Q$ be an auxiliary function such that
\begin{align*}
Q(\alpha, \alpha') &= \mathrm{E} \left[ \log p(\mathbf{k_1}, \mathbf{y} | \theta ) | \mathbf{k_1}, \theta' \right] \\
&= \mathrm{E} \left[ \sum_{i=1}^{N} \sum_{l=1}^{M} I(y(i)=l) (\log \phi_l(k_1(i)) + \log \alpha_l) | \mathbf{k_1}, \theta' \right] \\
&= \sum_{i=1}^{N} \sum_{l=1}^{M} \mathrm{E} \left[ I(y(i)=l) | \mathbf{k_1}, \theta' \right] (\log \phi_l(k_1(i)) + \log \alpha_l)  \\
&= \sum_{i=1}^{N} \sum_{l=1}^{M} p(y(i) = c | k_1(i), \theta') (\log \phi_l(k_1(i)) + \log \alpha_l)
\end{align*}

In the E-step of the EM algorithm, we compute $p(y(i) = c | k_1(i), \theta')$. 
In our case of $M=2$ the E-step is given by:
\begin{align*}
p(y(i) = 1 | k_1(i), \mathbf{\theta'}) &= \frac{\alpha \binom{n}{k_1(i)} p^{k_1(i)} (1-p)^{n-k_1(i)}}{\alpha \binom{n}{k_1(i)} p^{k_1(i)} (1-p)^{n-k_1(i)} + (1-\alpha) \binom{n}{k_1(i)} q^{k_1(i)} (1-q)^{n-k_1(i)}} \\
p(y(i) = 0 | k_1(i), \mathbf{\theta'}) &= \frac{(1-\alpha) \binom{n}{k_1(i)} q^{k_1(i)} (1-q)^{n-k_1(i)}}{\alpha \binom{n}{k_1(i)} p^{k_1(i)} (1-p)^{n-k_1(i)} + (1-\alpha) \binom{n}{k_1(i)} q^{k_1(i)} (1-q)^{n-k_1(i)}}
\end{align*}

In the M-step, we compute
\begin{align*}
\hat{\theta} &= \argmax_\theta Q(\alpha, \alpha') \\
&= \argmax_\theta \sum_{i=1}^{N} \sum_{l=1}^{M} p(y(i) = c | k_1(i), \theta') (\log \phi_l(k_1(i)) + \log \alpha_l)
\end{align*}

M-step:

$\alpha_1^{k+1} = \alpha^{k+1} = \frac{1}{N} \sum_{i=1}^{N} p(y(i) = 1 | k_1(i), \mathbf{\theta^{k}})$ \\
$\alpha_0^{k+1} = 1 - \alpha_1^{k+1} = 1 - \alpha^{k+1}$\\

